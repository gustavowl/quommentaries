\subsection{Section 6.1}

\subsubsection{Exercise 6.1}
This Exercise refers to Equation (6.5),
not to Equation (6.3), which describes the oracle's action.
Recall that the $\delta_{ij}$ notation means
\begin{align}
    \delta_{ij} = \begin{cases}
        1 & \text{if\ } i = j \\
        0 & \text{if\ } i \neq j
    \end{cases}.
\end{align}
Thus, Equation (6.5) essentially means that if
$x = 0$, then $\ket0 \rightarrow \ket0$, else,
$x \neq 0$ and $\ket x \rightarrow - \ket x$.
This is exactly the behaviour of $2 \ketbra{0}{0} - I$,
which is described in the following paragraphs.

First, the operator is easily verifiable to be unitary,
\begin{align}
    (2 \ketbra{0}{0} - I)^\dagger (2 \ketbra{0}{0} - I) &=
        (2 \ketbra{0}{0} - I)^2 \\
    &= 4 \ket0 \braket{0}{0} \bra0 - 2 \cdot 2 I \ketbra{0}{0} + I^2 \\
    &= 4 \ketbra00 - 4 \ketbra00 + I \\
    &= I.
\end{align}

Show that $(2 \ketbra00 - I) \ket0 = \ket0$,
\begin{align}
    (2 \ketbra{0}{0} - I) \ket0 &= 2 \ket{0} \braket{0}{0} - I \ket{0} \\
    &= 2 \ket0 - \ket0 \\
    &= \ket 0.
\end{align}
For any other state
    \footnote{Recall that Grover's Algorithm operates on the database's indices,
    and that they form an orthonormal basis, i.e. $\braket{i}{j} = \delta_{ij}$.}
$\ket x$, the phase is shifted,
\begin{align}
    (2 \ketbra{0}{0} - I) \ket x &= 2 \ket{0} \braket{0}{x} - I \ket{x} \\
    &= 0 \ket0 - \ket{x} \\
    &= - \ket{x}.
\end{align}

These actions can be easily inferred by analysing $(2 \ketbra{0}{0} - I)$'s matricial form,
\begin{align}
    \left[ \begin{matrix}
        2 & 0 & 0 & \cdots & 0 \\
        0 & 0 & 0 & \cdots & 0 \\
        0 & 0 & 0 & \cdots & 0 \\
        \vdots & \vdots & \vdots & \ddots & \vdots \\
        0 & 0 & 0 & \cdots & 0
    \end{matrix} \right]
    -
    \left[ \begin{matrix}
        1 & 0 & 0 & \cdots & 0 \\
        0 & 1 & 0 & \cdots & 0 \\
        0 & 0 & 1 & \cdots & 0 \\
        \vdots & \vdots & \vdots & \ddots & \vdots \\
        0 & 0 & 0 & \cdots & 1
    \end{matrix} \right]
    =
    \left[ \begin{matrix}
        1 & 0 & 0 & \cdots & 0 \\
        0 & -1 & 0 & \cdots & 0 \\
        0 & 0 & -1 & \cdots & 0 \\
        \vdots & \vdots & \vdots & \ddots & \vdots \\
        0 & 0 & 0 & \cdots & -1
    \end{matrix} \right].
\end{align}

\subsubsection{Equation 6.6}
In case it is not clear how this Equation was obtained,
the step by step solution is as follows,
\begin{align}
    H^{\otimes n} (2 \ketbra{0}{0} - I) H^{\otimes n} &=
        H^{\otimes n} (2 \ketbra{0}{0} H^{\otimes n} - H^{\otimes n}) \\
    &= 2 H^{\otimes n} \ketbra{0}{0} H^{\otimes n} - H^{\otimes n}) H^{\otimes n} \\
    &= 2 \ket{+}^{\otimes n} \bra{+}^{\otimes n} - I.
\end{align}
Then, using $\ket\psi$ as defined in Equation 6.4,
\begin{align}
    2 \ket{+}^{\otimes n} \bra{+}^{\otimes n} - I = 2 \ketbra{\psi}{\psi} - I.
\end{align}
Thus obtaining the desired result.

\subsubsection{Exercise 6.2}
Applying the operation to the general state,
\begin{align}
    (2 \ketbra{\psi}{\psi} - I) \sum_k \alpha_k \ket{k} = 
    \sum_k 2 \alpha_k \ket{\psi} \braket{\psi}{k} - \alpha_k \ket{k}.
\end{align}
Using Equation (6.4), and the definition of $\delta_{ij}$,
\begin{align}
    \sum_k 2 \alpha_k \ket{\psi} \braket{\psi}{k} - \alpha_k \ket{k} &=
    \sum_k 2 \alpha_k \frac{1}{N} \sum_{ij} \ket{i} \braket{j}{k} - \alpha_k \ket{k} \\
    & = \sum_k 2 \frac{\alpha_k}{N} \sum_{ij} \ket{i} \delta_{jk} - \alpha_k \ket{k} \\
    & = \sum_k 2 \frac{\alpha_k}{N} \sum_i \ket{i} - \alpha_k \ket{k}.
\end{align}
Note that $\sum_k \alpha_k/N$ is a constant
(written as $\langle\alpha\rangle$). Thus,
\begin{align}
    \sum_k 2 \frac{\alpha_k}{N} \sum_i \ket{i} - \alpha_k \ket{k} =
    2 \langle\alpha\rangle \sum_i \ket{i} - \sum_k \alpha_k \ket{k}.
\end{align}
Since $\ket{k}$ and $\ket{i}$ correspond to the same orthonormal basis,
it is possible to rename and rearrange
\begin{align}
    2 \langle\alpha\rangle \sum_i \ket{i} - \sum_k \alpha_k \ket{k} &=
    \sum_k 2 \langle\alpha\rangle \ket{k} - \sum_k \alpha_k \ket{k} \\
    &= \sum_k \left[ 2 \langle\alpha\rangle - \alpha_k \right] \ket{k}.
\end{align}
Thus obtaining the desired answer.

\subsubsection{Notes on Grover's Algorithm}
\label{sec:nielsen-and-chuang-notes-on-grovers-algorithm}
\todo{Notes on finding $\ket0$ index}
\begin{comment}
If the searched item is \emph{not} in the database,
then applying the Grover Iteration should have no effects.
In fact, the conditional shift is partially responsible for this behaviour.
For instance,this is exactly what happens and exactly what is described by
$2 \ketbra{0}{0} - I$, applying it to $\ket 0$,

The first step of Grover's Algorithm is to apply
$H^{\otimes n}$ to the initial state $\ket0^{\otimes n}$.
Thus, the oracle will compute all the possible results due to Quantum Superposition
(Quantum Parallelism).
Suppose that the action of the oracle is none.
That is, the searched value is \emph{not} in the database.
Then, $O H^{\otimes n} \ket0^{\otimes n} = H^{\otimes n} \ket0^{\otimes n}$.
Since the first action after the oracle is to apply $H^{\otimes n}$ again,
$H^{\otimes n} H^{\otimes n} \ket0^{\otimes n} = \ket0^{\otimes n}$.

On the other hand, if the oracle changes the $H^{\otimes n} \ket0^{\otimes n}$ state,
then the searched item is in the database.

refer back to matricial form of conditional phase shift? to when
$(2 \ketbra{0}{0})\ket0$ and $(2 \ketbra{0}{0})(- \ket0)$?

Note that the conditional phase shift does not flip $- \ket 0$:
$(2 \ketbra00 - I)(- \ket0) 0 -2 \ket0 + \ket0 = - \ket0$.

step-by-step example?
\end{comment}