\subsection{Section 6.1}

\subsubsection{Exercise 6.1}
This Exercise refers to Equation (6.5),
not to Equation (6.3), which describes the oracle's action.
Recall that the $\delta_{ij}$ notation means
\begin{align}
    \delta_{ij} = \begin{cases}
        1 & \text{if\ } i = j \\
        0 & \text{if\ } i \neq j
    \end{cases}.
\end{align}
Thus, Equation (6.5) essentially means that if
$x = 0$, then $\ket0 \rightarrow \ket0$, else,
$x \neq 0$ and $\ket x \rightarrow - \ket x$.
This is exactly the behaviour of $2 \ketbra{0}{0} - I$,
which is described in the following paragraphs.

First, the operator is easily verifiable to be unitary,
\begin{align}
    (2 \ketbra{0}{0} - I)^\dagger (2 \ketbra{0}{0} - I) &=
        (2 \ketbra{0}{0} - I)^2 \\
    &= 4 \ket0 \braket{0}{0} \bra0 - 2 \cdot 2 I \ketbra{0}{0} + I^2 \\
    &= 4 \ketbra00 - 4 \ketbra00 + I \\
    &= I.
\end{align}

Show that $(2 \ketbra00 - I) \ket0 = \ket0$,
\begin{align}
    (2 \ketbra{0}{0} - I) \ket0 &= 2 \ket{0} \braket{0}{0} - I \ket{0} \\
    &= 2 \ket0 - \ket0 \\
    &= \ket 0.
\end{align}
For any other state
    \footnote{Recall that Grover's Algorithm operates on the database's indices,
    and that they form an orthonormal basis, i.e. $\braket{i}{j} = \delta_{ij}$.}
$\ket x$, the phase is shifted,
\begin{align}
    (2 \ketbra{0}{0} - I) \ket x &= 2 \ket{0} \braket{0}{x} - I \ket{x} \\
    &= 0 \ket0 - \ket{x} \\
    &= - \ket{x}.
\end{align}

These actions can be easily inferred by analysing $(2 \ketbra{0}{0} - I)$'s matricial form,
\begin{align}
    \left[ \begin{matrix}
        2 & 0 & 0 & \cdots & 0 \\
        0 & 0 & 0 & \cdots & 0 \\
        0 & 0 & 0 & \cdots & 0 \\
        \vdots & \vdots & \vdots & \ddots & \vdots \\
        0 & 0 & 0 & \cdots & 0
    \end{matrix} \right]
    -
    \left[ \begin{matrix}
        1 & 0 & 0 & \cdots & 0 \\
        0 & 1 & 0 & \cdots & 0 \\
        0 & 0 & 1 & \cdots & 0 \\
        \vdots & \vdots & \vdots & \ddots & \vdots \\
        0 & 0 & 0 & \cdots & 1
    \end{matrix} \right]
    =
    \left[ \begin{matrix}
        1 & 0 & 0 & \cdots & 0 \\
        0 & -1 & 0 & \cdots & 0 \\
        0 & 0 & -1 & \cdots & 0 \\
        \vdots & \vdots & \vdots & \ddots & \vdots \\
        0 & 0 & 0 & \cdots & -1
    \end{matrix} \right].
\end{align}

\subsubsection{Equation 6.6}
\label{sec:nielsen-and-chuang-equation-6-6}
In case it is not clear how this Equation was obtained,
the step by step solution is as follows,
\begin{align}
    H^{\otimes n} (2 \ketbra{0}{0} - I) H^{\otimes n} &=
        H^{\otimes n} (2 \ketbra{0}{0} H^{\otimes n} - H^{\otimes n}) \\
    &= 2 H^{\otimes n} \ketbra{0}{0} H^{\otimes n} - H^{\otimes n} H^{\otimes n} \\
    &= 2 \ket{+}^{\otimes n} \bra{+}^{\otimes n} - I.
\end{align}
Then, using $\ket\psi$ as defined in Equation 6.4,
\begin{align}
    2 \ket{+}^{\otimes n} \bra{+}^{\otimes n} - I = 2 \ketbra{\psi}{\psi} - I.
\end{align}
Thus obtaining the desired result.

\subsubsection{Exercise 6.2}
Applying the operation to the general state,
\begin{align}
    (2 \ketbra{\psi}{\psi} - I) \sum_k \alpha_k \ket{k} = 
    \sum_k 2 \alpha_k \ket{\psi} \braket{\psi}{k} - \alpha_k \ket{k}.
\end{align}
Using Equation (6.4), and the definition of $\delta_{ij}$,
\begin{align}
    \sum_k 2 \alpha_k \ket{\psi} \braket{\psi}{k} - \alpha_k \ket{k} &=
    \sum_k 2 \alpha_k \frac{1}{N} \sum_{ij} \ket{i} \braket{j}{k} - \alpha_k \ket{k} \\
    & = \sum_k 2 \frac{\alpha_k}{N} \sum_{ij} \ket{i} \delta_{jk} - \alpha_k \ket{k} \\
    & = \sum_k 2 \frac{\alpha_k}{N} \sum_i \ket{i} - \alpha_k \ket{k}.
\end{align}
Note that $\sum_k \alpha_k/N$ is a constant
(written as $\langle\alpha\rangle$). Thus,
\begin{align}
    \sum_k 2 \frac{\alpha_k}{N} \sum_i \ket{i} - \alpha_k \ket{k} =
    2 \langle\alpha\rangle \sum_i \ket{i} - \sum_k \alpha_k \ket{k}.
\end{align}
Since $\ket{k}$ and $\ket{i}$ correspond to the same orthonormal basis,
it is possible to rename and rearrange
\begin{align}
    2 \langle\alpha\rangle \sum_i \ket{i} - \sum_k \alpha_k \ket{k} &=
    \sum_k 2 \langle\alpha\rangle \ket{k} - \sum_k \alpha_k \ket{k} \\
    &= \sum_k \left[ 2 \langle\alpha\rangle - \alpha_k \right] \ket{k}.
\end{align}
Thus obtaining the desired answer.

\subsubsection{Notes About Grover's Algorithm}
\label{sec:nielsen-and-chuang-notes-on-grovers-algorithm}
At a first glance,
it may appear that Grover's Algorithm cannot find a value if its index is 0.
This erroneous thought may come to one's mind due to the phase shift step;
when $\ket0$ is left unchanged.
To show that Grover's Algorithm works perfectly,
two scenarios will be considered:
the searched value is \emph{not} in the ``database'';
the searched value has index 0 in the ``database''.
It can be seen that, for both scenarios, the obtained result is different.

\paragraph{Value not found.}
To leave $\ket0$ unchanged is important in case
the searched value is not in the ``database''.
To illustrate this scenario, let the Grover iteration be denoted by
\begin{align}
    H^{\otimes n} P_h H^{\otimes n} O \ket\varphi,
\end{align}
where $P_h$ is the phase shift, $O$ is the oracle.
Let $\ket\varphi = \ket\psi$, the state described in Equation 6.4 --
immediately before the first Grover Iteration.
Thus,
\begin{align}
    H^{\otimes n} P_h H^{\otimes n} O \ket\psi &=
        H^{\otimes n} P_h H^{\otimes n} \ket\psi \\
    &= H^{\otimes n} P_h \ket{0}^{\otimes n} \\
    &= H^{\otimes n} \ket{0}^{\otimes n} \\
    &= \ket\psi.
\end{align}
Therefore, it is clear that after $\sqrt{n}$ Grover iterations,
the initial state $\ket\psi$ will remain unchanged if
the desired value is not in the ``database''.

Nevertheless, note that the obtained result is the same if
the Grover iteration was denoted with the aid of
\hyperref[sec:nielsen-and-chuang-equation-6-6]{Equation 6.6},
\begin{align}
    (2 \ketbra{\psi}{\psi} - I) O \ket\psi &= (2 \ketbra{\psi}{\psi} - I) \ket\psi \\
    &= 2 \ket\psi \braket{\psi}{\psi} - \ket\psi \\
    &= \ket\psi
\end{align}

\paragraph{Value in index 0.}
If the desired value is in the ``database'', and in the first index, i.e. 0,
the Algorithm will also work perfectly.
In this scenario, the first Grover iteration will perform the following action,
\begin{align}
    H^{\otimes n} P_h H^{\otimes n} O \ket\psi &=
        H^{\otimes n} P_h H^{\otimes n} \left(- \frac{2\ket{0}^{\otimes n}}{\sqrt{2}^n} + \ket\psi \right) \\
    &= H^{\otimes n} P_h \left(- \frac{2\ket\psi}{\sqrt{2}^n} + \ket{0}^{\otimes n} \right) \\
    &= H^{\otimes n} \left(\frac{2}{\sqrt{2}^n} \left(\ket\psi -
        \frac{2}{\sqrt{2}^n} \ket{0}^{\otimes n} \right) + \ket{0}^{\otimes n} \right) \\
    &= H^{\otimes n} \left( \frac{2}{\sqrt{2}^n}\ket\psi -\frac{4}{2^n}\ket{0}^{\otimes n} +
        \ket{0}^{\otimes n} \right) \\
    &= H^{\otimes n} \left( \frac{1}{\sqrt{2}^{n-2}}\ket\psi +
        \frac{2^{n-2} - 1}{2^{n-2}}\ket{0}^{\otimes n} \right) \\
    &= \frac{1}{\sqrt{2}^{n-2}}\ket{0}^{\otimes n} + \frac{2^{n-2} - 1}{2^{n-2}}\ket\psi.
\end{align}

The same result would be obtained if \hyperref[sec:nielsen-and-chuang-equation-6-6]{Equation 6.6}
was used to denote part of the Grover iteration,
\begin{align}
    (2\ketbra{\psi}{\psi} - I)O \ket\psi &=
        (2\ketbra{\psi}{\psi} - I) \left(- \frac{2\ket{0}^{\otimes n}}{\sqrt{2}^n} + \ket\psi \right) \\
    &= - \frac{4 \ket\psi \braket{\psi}{0}^{\otimes n}}{\sqrt{2}^n} + 2 \ket\psi +
        \frac{2 \ket{0}^{\otimes n}}{\sqrt{2}^n} - \ket\psi \\
    &= - \frac{4}{2^n}\ket\psi + \ket\psi + \frac{2 \ket{0}^{\otimes n}}{\sqrt{2}^n} \\
    &= - \frac{1}{2^{n - 2}}\ket\psi + \ket\psi + \frac{1}{\sqrt{2}^{n - 2}} \ket{0}^{\otimes n} \\
    &= \frac{2^{n - 2} - 1}{2^{n - 2}}\ket\psi + \frac{1}{\sqrt{2}^{n - 2}} \ket{0}^{\otimes n}.
\end{align}

\paragraph{Conclusion.}
Note that the obtained result for both scenarios is different.
In the first, the original state is left unchanged.
In the second, the original state is changed;
though the obtained result may seem odd.
In order to properly comprehend the result in the latter scenario,
it may be useful to understand the geometric interpretation of
the Grover iteration.


\todo{notes about the geometric interpretation}