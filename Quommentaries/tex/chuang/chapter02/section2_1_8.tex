\subsection{Section 2.1.8}
\subsubsection{Exercise 2.35}
\label{sec:nielsen-and-chuang-exercise-2-35}
    In order to solve this exercise,
    it is necessary to find a spectral decomposition for $\vec{v} \vec{\sigma}$.
    Then, it is possible to apply the definition of Operator functions.
    
    With the aid of \hyperref[sec:nielsen-and-chuang-exercise-2-60]{Exercise 2.60},
    the required spectral decomposition is obtained:
    \begin{align}
        \vec{v}\vec{\sigma} &= +1 P_+ -1 P_- \\
        &= \frac{I + \vec{v}\vec{\sigma}}{2} - \frac{I - \vec{v}\vec{\sigma}}{2}
    \end{align}
    
    Now, calculating the value of $exp(i \theta \vec{v}\cdot\vec{\sigma})$
    and applying the definition of Operator functions:
    \begin{align}
        exp(i \theta \vec{v}\cdot\vec{\sigma}) &=
            exp(i \theta P_+) + exp(-i \theta P_-) \\
        &= exp(i \theta) P_+ + exp(-i \theta) P_- \\
        &= e^{i \theta} P_+ + e^{-i \theta} P_-
    \end{align}
    
    Then, applying Euler's Formula:
    \begin{align}
        exp(i \theta \vec{v}\cdot\vec{\sigma}) &=
            cos(\theta)P_+ + i sin(\theta)P_+ +
            cos(\theta)P_- - i sin(\theta)P_- \\
        &= cos(\theta)\frac{I + \vec{v}\vec{\sigma}}{2} +
            i sin(\theta)\frac{I + \vec{v}\vec{\sigma}}{2} +
            cos(\theta)\frac{I - \vec{v}\vec{\sigma}}{2} -
            i sin(\theta)\frac{I - \vec{v}\vec{\sigma}}{2} \\
        &= cos(\theta) \left(
                \frac{I + \vec{v}\vec{\sigma} + I - \vec{v}\vec{\sigma}}{2}
            \right)
            + i sin(\theta) \left(
                \frac{I + \vec{v}\vec{\sigma} - I + \vec{v}\vec{\sigma}}{2}
            \right) \\
        &= cos(\theta) I + i\ sin(\theta) \vec{v}\vec{\sigma}
    \end{align}
    
    Thus obtaining the required result.

%%%%%%%%%%%%%%%%%%%%%%%%%%%%%%%%%%%%%%%%%%%%%%%%%%%%%%%%%%%%%%%%%%%%%%%%%%%%
\subsubsection{Equation 2.60}
\label{sec:nielsen-and-chuang-equation-2-60}
    It is known that $tr(U A U^\dagger) = tr(A)$. Therefore,
    \begin{align}
        tr(A \ketbra{\psi}{\psi}) = tr(U A \ketbra{\psi}{\psi} U^\dagger))
    \end{align}
    
    By Equation 2.22, $\sum_i \ketbra{i}{i} = I$.
    Since $I$ is an Unitary Operator, it is possible to write
    \begin{align}
        tr(U A \ketbra{\psi}{\psi} U^\dagger) &=
            tr(I A \ketbra{\psi}{\psi} I^\dagger)) \\
        &= tr(I A \ketbra{\psi}{\psi} I)) \\
        &= tr(\sum_{ij} \ketbra{i}{i} A \ketbra{\psi}{\psi} \ketbra{j}{j})
    \end{align}
    
    Since $\bra{i}A\ket{\psi}$ and $\braket{\psi}{j}$ are scalars,
    \begin{align}
        tr(\sum_{ij} \ketbra{i}{i} A \ketbra{\psi}{\psi} \ketbra{j}{j}) &=
            tr(\sum_{ij} \bra{i}A\ket{\psi} \braket{\psi}{j} \ketbra{i}{j})
    \end{align}
    
    Similarly to Equation 2.25,
    $tr(\sum_{ij} \bra{i}A\ket{\psi} \braket{\psi}{j} \ketbra{i}{j})$
    is an Outer Product representation \\
    for $A\ketbra{\psi}{\psi}$
    where element $m_{ij} = \bra{i}A\ket{\psi} \braket{\psi}{j}$
    (check Section \ref{sec:nielsen-and-chuang-outer-product-of-a} for details).
    Then, by the definition of trace (Equation 2.59):
    \begin{align}
        tr(\sum_{ij} \bra{i}A\ket{\psi} \braket{\psi}{j} \ketbra{i}{j}) &=
            m_{ii} \\
        &= \sum_i \bra{i}A\ket{\psi} \braket{\psi}{i}
    \end{align}
    
\subsubsection[Trace of ketbra equals braket]
{$tr(\ketbra{\psi}{\varphi} = \braket{\varphi}{\psi})$}
\label{sec:nielsen-and-chuang-trace-of-ketbra-equals-braket}
    I decided to add this section because this equation is used
    throughout the book, e.g. \hyperref[sec:nielsen-and-chuang-equations-2-208-209]
    {Equation 2.208 to 2.209}, and
    \hyperref[sec:nielsen-and-chuang-exercise-2-82]{Exercise 2.82}.
    I do not remember it being explicitly stated or explained, though.
    
    Suppose two different states $\ket{\psi}$ and $\ket{\varphi}$. Then,
    following the same reasoning as \hyperref[sec:nielsen-and-chuang-equation-2-60]
    {Equation 2.60} and Equation 2.61:
    \begin{align}
        tr(\braket{\psi}{\varphi}) &= tr(I\braket{\psi}{\varphi}) \\
        &= \sum_i \bra{i} I \ket{\psi} \braket{\varphi}{i} \\
        &= \sum_i \braket{i}{\psi} \braket{\varphi}{i} \\
        &= \sum_i \bra{\varphi} \ketbra{i}{i} \ket{\psi} \\
        &= \bra{\varphi} I \ket{\psi} \\
        &= \braket{\varphi}{\psi}
    \end{align}
    
    Note that although \hyperref[sec:nielsen-and-chuang-equation-2-60]
    {Equation 2.60} defines a orthonormal basis $\ket{i}$ containing
    $\ket{\psi}$, this is not necessary.
    The only restriction is that, if $\ket{\psi} \in V$ and
    $\ket{\varphi} \in W$, then $dim(V) = dim(W)$.
    In other words, that $\ketbra{\psi}{\varphi}$ is a square matrix.